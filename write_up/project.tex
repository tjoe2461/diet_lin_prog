\documentclass[12pt, english]{report}
\usepackage[utf8]{inputenc}
\usepackage[english]{babel}
\usepackage[]{amsthm}
\usepackage[]{graphicx}
\usepackage[]{amsmath}
\usepackage[]{amssymb}
\usepackage[]{tikz-cd}
\usepackage[]{esdiff}
\usepackage[]{physics}
\usepackage[bottom]{footmisc}
\usepackage[]{biblatex}
\usepackage[letterpaper]{geometry}
\usepackage[]{float}
\usepackage[Glenn]{fncychap}
\usepackage[]{multicol}
\usepackage[font=small]{caption}
\usepackage{xcolor,colortbl}
\usepackage[]{multirow}
\usepackage[]{hyperref}
\usepackage[]{booktabs}
\usepackage[]{gensymb}
\usepackage[]{fancyhdr}
\addbibresource{project.bib}
\geometry{top=1.0in, bottom=1.0in, left=1.0in, right=1.0in}
\title{Synergizing Cross-Departmental Bio-Optimization Frameworks to Actualize Aspirational Wellness Vectors through Innovative Caloric Recalibration Methodologies and Paradigm-Shifting Nutritional Engagement Strategies in a Dynamic, Future-Proofed Ecosystem of Holistic Health Empowerment\footnote{Optimization of core competencies was achieved through the implementation of next-generation ideation frameworks, facilitated by advanced language models to drive innovation and maximize stakeholder value proposition (Claude 3.5 Sonnet).}}
\author{Damian Jadczak, Tyler Mitchell and Jacob Molnia}
\date\today
\definecolor{cadmiumgreen}{rgb}{0.0, 0.42, 0.24}
\hypersetup{
    colorlinks=true,
    linkcolor=cadmiumgreen,
    filecolor=cadmiumgreen,      
    urlcolor=cadmiumgreen,
    citecolor=cadmiumgreen,
    pdftitle={Synergizing Cross-Departmental Bio-Optimization Frameworks to Actualize Aspirational Wellness Vectors through Innovative Caloric Recalibration Methodologies and Paradigm-Shifting Nutritional Engagement Strategies in a Dynamic, Future-Proofed Ecosystem of Holistic Health Empowerment},
    pdfpagemode=FullScreen,
}
\begin{document}
\maketitle
\tableofcontents
\pagebreak
\chapter{Introduction}
\vspace*{-1in}
\section{Objective}
Our objective is to formulate a linear program to minimize the grocery costs of our group while maintaining nutritional recommendations and preferences. With this information, we will further algorithmically generate a weekly diet for each member based on their needs.
\chapter{Methodology}
\vspace*{-1in}
\section{Diets}
We begin formulating our linear program by agreeing on the dietary restrictions for each group member and their dietary goals\footnote{We calculated our dietary requirements using the \href{https://apps.apple.com/us/app/stupid-simple-macro-tracker/id1210995590}{Stupid Simple Macro Tracker}.}:
\begin{table}[H]
    \centering
    \caption[short]{Weekly dietary requirements.}
    \begin{tabular}{cccc}
        \toprule
        Diets & Damian & Tyler & Jacob\\ 
        \midrule
        Protein & 204 &312 & 222 \\ 
        Fats &27 &56 & 49\\ 
        Carbohydrates &143 &187 & 222 \\
    \bottomrule
    \label{tab:diet}
    \end{tabular}
\end{table}
\section{Dataset}
We obtained our data by modifying an existing set from Tirthajyoti Sarkar's \href{https://github.com/tirthajyoti/Optimization-Python/blob/master/Data/diet.xls}{Optimization-Python} project under MIT licensing.
\pagebreak
\printbibliography
\end{document}