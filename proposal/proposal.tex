\documentclass[12pt, english]{article}
\usepackage[utf8]{inputenc}
\usepackage[english]{babel}
\usepackage[]{amsthm}
\usepackage[]{graphicx}
\usepackage[]{amsmath}
\usepackage[]{amssymb}
\usepackage[]{tikz-cd}
\usepackage[]{esdiff}
\usepackage[]{physics}
\usepackage[bottom]{footmisc}
\usepackage[]{biblatex}
\usepackage[letterpaper]{geometry}
\usepackage[]{float}
\usepackage[Glenn]{fncychap}
\usepackage[]{multicol}
\usepackage[font=small]{caption}
\usepackage{xcolor,colortbl}
\usepackage[]{multirow}
\usepackage[]{hyperref}
\usepackage[]{booktabs}
\usepackage[]{gensymb}
\usepackage[]{fancyhdr}
\usepackage{setspace}
\onehalfspacing
\geometry{top=1.0in, bottom=1.0in, left=1.0in, right=1.0in}
\title{MA 3231 Project Proposal}
\author{Damian Jadczak, Tyler Mitchell, Jacob Molnia}
\date\today
\begin{document}
\maketitle
\paragraph{Intention}
In our Diet Problem Project, we want to find the cheapest food options for our entire group (imagining we live together). We will analyze food items that you can buy from the store, minimizing the cost of food items whilst maintaining daily nutritional recommendations within our preferences. Then, we will create a weekly diet plan for each group member. 
\paragraph{Plan}
Our intention is to research and analyze 30 food items (subject to change), looking at their cost (in USD) and nutritional content. Then, we will look at the daily recommend nutritional intake (calories, carbohydrates, etc.) for each group member depending on factors like height, weight, and lifestyle. Next, we will analyze the daily upper limit for unhealthy content (saturated fats, cholesterol, etc.) for each group member considering the factors described above. Next, we will factor in our own preferences for food, creating additional constraints to work with. 
\paragraph{Program}
We will form Linear Programs minimizing the cost of ingredients (USD) that adhere to each of our own nutritional requirements and limits. We will find optimal feasible solutions to our linear programs using MATLAB or Python. After this, we will combine the optimal feasible solutions of each of our linear programs, getting the net cost (USD).  
\paragraph{Conclusion}
Once we get our results, we will find meals that we can make with the ingredients and implement them into the weekly diet plans of each group member. We will try to create diversity in meals and consider how realistic our diet plans are. Then, we will see if there are any changes that we could make that could further reduce costs.  
\end{document}